\documentclass[12pt]{article}
 \usepackage[margin=1in]{geometry} 
\usepackage{amsmath,amsthm,amssymb,amsfonts}
 
\newcommand{\N}{\mathbb{N}}
\newcommand{\Z}{\mathbb{Z}}
 
\newenvironment{problem}[2][Problem]{\begin{trivlist}
\item[\hskip \labelsep {\bfseries #1}\hskip \labelsep {\bfseries #2.}]}{\end{trivlist}}
%If you want to title your bold things something different just make another thing exactly like this but replace "problem" with the name of the thing you want, like theorem or lemma or whatever
 
\begin{document}
 
%\renewcommand{\qedsymbol}{\filledbox}
%Good resources for looking up how to do stuff:
%Binary operators: http://www.access2science.com/latex/Binary.html
%General help: http://en.wikibooks.org/wiki/LaTeX/Mathematics
%Or just google stuff
 
\title{Chapter 1 - The Real and Complex Number Systems}
\date{}
\maketitle
 
\begin{problem}{1}
If $r$ is rational ($r\neq 0$) and $x$ is irrational, prove that $r+x$ and $rx$ are irrational.
\end{problem}
 
\begin{proof}
Write $r = \frac{m}{n}$, where $m$ and $n$ are nonzero integers.  Suppose $r+x$ were rational.  Then there exists integers $p$ and $q$, with $q\neq 0$ such that
\begin{equation}
r + x = \frac{p}{q}
\end{equation}
Then $x$ can be expressed as
\begin{equation}
x = \frac{p}{q} - \frac{m}{n} = \frac{pn - mq}{qn} \in \mathbb{Q}
\end{equation}
contradicting our assumption about $x$.  So $r+x$ must be irrational.

\vspace{2mm}

Now, suppose $rx$ were rational.  Then there exists integers $p$ and $q$, with $q\neq0$ such that
\begin{equation}
x = \frac{p}{q}\cdot\frac{n}{m}
\end{equation}
since $m\neq 0$.  Then $x$ can be written as
\begin{equation}
x = \frac{pn}{qm} \in \mathbb{Q}
\end{equation}
contradicting our assumption about $x$.  So $rx$ must be irrational.
\end{proof}

\begin{problem}{2}
Prove that there is no rational number whose square is 12.
\end{problem}
\begin{proof}
Suppose there exists a rational number $r \in \mathbb{Q}$ such that $r^2 = 12$.  We can write $r = \frac{m}{n}$ where $m$ and $n$ share no common factors.  Then
\begin{equation}\label{Problem2-1}
\frac{m^2}{n^2} = 3\cdot 4 \implies m^2 = 3\cdot 4n^2
\end{equation}
Thus $m^2$ is divisible by 3.  This implies $m$ is divisible by 3 (otherwise $m^2$ would not be).  Hence, $m^2$ is divisible by 9, and so is the right hand side of (\ref{Problem2-1}).  This implies that $4n^2$ is divisible by 3.  Since 4 is not divisible by 3, it follows that $n^2$, and thus $n$ is divisible by 3.  This contradicts the fact that $m$ and $n$ share no common factors.  Thus, there can be no rational number that satisfies $r^2 = 12$.
\end{proof}

\begin{problem}{3}
Prove Proposition 1.15: The axioms for multiplication in a field imply the following statements.

(a) If $x\neq 0$ and $xy = xz$ then $y = z$

(b) If $x\neq 0$ and $xy = x$ then $y = 1$

(c) If $x\neq 0$ and $xy = 1$ then $y = 1/x$

(d) If $x\neq 0$ then $1/(1/x) = x$.

\end{problem}
\begin{proof}
(a) \begin{equation}
y = (1/x)xy = (1/x)xz = 1z = z
\end{equation}

(b) Take $z = 1$ in (a)

(c) Take $z = 1/x$ in (a)

(d) This follows from (c) if we replace $x$ with $1/x$ and $y$ with $x$
\end{proof}

\begin{problem}{4}
Let $E$ be a nonempty subset of an ordered set; suppose $\alpha$ is a lower bound of $E$ and $\beta$ is an upper bound of $E$.  Prove that $\alpha\leq\beta$.
\end{problem}
\begin{proof}
Suppose $\alpha > \beta$.  Let $x\in E$; since $\alpha$ is a lower bound of $E$, we must have $\alpha \leq x$.  Since $\alpha > \beta$ and $>$ is transitive, it then follows that $\beta < x$.  But this contradicts the fact that $\beta$ is an upper bound of $E$.  So $\alpha > \beta$ must be false, i.e. $\alpha \leq \beta$.
\end{proof}
\begin{problem}{5}
Let $A$ be a nonempty set of real numbers which is bounded below.  Let $-A$ be the set of all numbers $-x$, where $x\in A$.  Prove that
\begin{equation}\label{infsup}
\inf A = -\sup(-A)
\end{equation}
\end{problem}
\begin{proof}
Since $A$ is bounded below, $\gamma = \inf A$ exists in $\mathbb{R}$, and $\gamma\leq x$ for all $x \in A$.  This implies $-\gamma \geq -x$ for all $x \in A$, or $-\gamma \geq y$ for all $y \in -A$.  So $-\gamma$ is an upper bound of $-A$.  Let $\kappa < -\gamma$, then $-\kappa > \gamma$, so that $-\kappa$ is not a lower bound of $A$.  Hence, there exists $x \in A$ such that $-\kappa > x$ or $\kappa < -x \in -A$.  Hence $\kappa$ is not an upper bound of $-A$.  Then by definition, $-\gamma$ is the supremum of $-A$, i.e
\begin{equation}
-\inf A = \sup(-A)
\end{equation}
which is equivalent to (\ref{infsup})
\end{proof}
\begin{problem}{6}
\end{problem}
\begin{problem}{7}

\end{problem}
\begin{problem}{8}
Prove that no order can be defined in the complex field that turns it into an ordered field.  \emph{Hint}: $-1$ is a square.
\end{problem}
\begin{proof}
Let $>$ be an order on $\mathbb{C}$.  Assume this turns $\mathbb{C}$ into an ordered field.  Since $i \neq 0$, we must either have $i > 0$ or $i < 0$.  First, assume $i > 0$.  Then $-1 = i^2 > 0$.  Add 1 to both sides to obtain $0 > 1$.  But $1 = (-1)(-1) > 0$ since $-1$ is positive, resulting in a contradiction.  Now assume $i < 0$.  Add $-i$ to both sides to obtain $0 < -i$.  Hence $-1 = (-i)^2 > 0$.  Again, we have $0 > 1$, but $1 = (-1)(-1) > 0$ resulting in another contradiction.  So $\mathbb{C}$ cannot be an ordered field under this order.
\end{proof}
\begin{problem}{9}

\end{problem}
\begin{problem}{10}

\end{problem}
\begin{problem}{11}

\end{problem}
\begin{problem}{12}

\end{problem}
\begin{problem}{13}

\end{problem}
\begin{problem}{14}

\end{problem}
\begin{problem}{15}

\end{problem}
\begin{problem}{16}
(Incomplete) Suppose $k \geq 3$, $\mathbf{x},\mathbf{y} \in \mathbb{R}^k$, $|\mathbf{x}-\mathbf{y}| = d > 0$, and $r > 0$.  Prove:

(a) If $2r > d$, there are infinitely many $\mathbf{z}\in \mathbb{R}^k$ such that
\begin{equation}
|\mathbf{z} - \mathbf{x}| = |\mathbf{z} - \mathbf{y}| = r
\end{equation}

(b) If $2r = d$, there is exactly one such $\mathbf{z}$.

(c) If $2r < d$, there is no such $\mathbf{z}$.

How must these statements be modified if $k$ is 2 or 1?
\end{problem}
\begin{proof}
(a) Since $2r > d$ it follows that $r^2 - (d/2)^2 > 0$.  Then let $\epsilon = \sqrt{r^2 - (d/2)^2} > 0$.  Suppose there is a vector $\mathbf{q} \in \mathbb{R}^k$ such that $|\mathbf{q}| = \epsilon$ and $\mathbf{q} \cdot (\mathbf{y} - \mathbf{x} )= 0$.  Then $\mathbf{z} = \mathbf{x} + \frac{1}{2}(\mathbf{y} - \mathbf{x}) + \mathbf{q}$ will meet the desired requirements: expanding out $|\mathbf{z} - \mathbf{x}|^2 = (\mathbf{z} - \mathbf{x}) \cdot (\mathbf{z} - \mathbf{x})$ and $|\mathbf{z} - \mathbf{y}|^2 = (\mathbf{z} - \mathbf{y}) \cdot (\mathbf{z} - \mathbf{y})$ and using the fact that $\mathbf{q}\cdot \mathbf{y} = \mathbf{q} \cdot \mathbf{x}$ we obtain the identical expression
\begin{align}
\frac{1}{4}(\mathbf{y} - \mathbf{x})\cdot (\mathbf{y} - \mathbf{x}) + \mathbf{q}\cdot \mathbf{q} = \left(\frac{d}{2}\right)^2 + \epsilon^2 = r^2
\end{align}
as desired.  We now must prove that there are infinitely many such $\mathbf{q}$.  

Since $|\mathbf{x} - \mathbf{y} | > 0$, we have $\mathbf{x} \neq \mathbf{y}$ and so there exists an integer $j$ satisfying $1\leq j \leq k$ such that $y_j - x_j \neq 0$.  Choose two other distinct integers $i, p$ that satisfy $1 \leq i,p \leq k$, and $i,p \neq j$ (this is possible since $k\geq 3$).\par
Let $\alpha = (y_j - x_j) \neq 0$, $\beta = (y_i - x_i)$ and $\gamma = (y_p - x_p)$.  Let $q_p\in \mathbb{R}$ satisfying:
\begin{equation}\label{qp}
|q_p| \leq \epsilon \left( \frac{\beta^2 + \alpha^2}{\gamma^2 + \beta^2 + \alpha^2}\right)^{1/2}
\end{equation}
The number on the right hand side of equation (\ref{qp}) is well-defined and positive, hence, there are infinitely many such $q_p$.\par
For some such $q_p$, consider the quadratic equation in the variable $q_i$:
\begin{equation}\label{quadratic}
q_i^2\left(\frac{\beta^2 + \alpha^2}{\alpha^2}\right) + q_i\frac{2q_p\beta\gamma}{\alpha^2} + q_p^2\frac{\gamma^2 + \alpha^2}{\alpha^2} - \epsilon^2 = 0
\end{equation}
There is at least one solution to this equation if the discriminant is non-negative.  After expanding out the discriminant and some algebra, it can be shown that the discriminant is non-negative if and only if (\ref{qp}) holds.  Thus at least one number $q_i$ satisfies this equation; let $q_i$ take this value. \par
Finally, define $q_j$ as:
\begin{equation}\label{qj}
 q_j = \frac{-1}{\alpha}(\beta q_i + \gamma q_p)  
 \end{equation}
 Let $\mathbf{q} \in \mathbb{R}^k$ be a vector satisfying:
\begin{align}
&\mathbf{q}_i = q_i \\
&\mathbf{q}_p = q_p \\
&\mathbf{q}_j = q_j \\
&\mathbf{q}_n = 0 \hspace{3mm} \text{if} \ n\neq i,p,j
\end{align}
Then $\mathbf{q} \cdot (\mathbf{y} - \mathbf{x}) = 0$ by appealing to equation (\ref{qj}) and the definition of $\gamma$, $\alpha$, and $\beta$.  Furthermore, the equation
\begin{equation}
|\mathbf{q}|^2 = q_i^2 + q_j^2 + q_p^2 = \epsilon^2
\end{equation}
is shown to be equivalent to equation (\ref{quadratic}) after plugging in expression (\ref{qj}) for $q_j$ and some algebra.  By construction, this equation holds and so we have $|\mathbf{q}| = \epsilon$.  Thus $\mathbf{q}$ satisfies the properties given at the start of the proof.  Since there are infinitely many $q_p$ that satisfy equation (\ref{qp}) there are infinitely many such $\mathbf{q}$ and the result follows.

\vspace{5mm}

(b) (Solve problem 13 first)


\end{proof}
\begin{problem}{17}

\end{problem}
\begin{problem}{18}
If $k \geq 2$ and $\mathbf{x} \in \mathbb{R}^k$, prove that there exists $\mathbf{y} \in \mathbb{R}^k$ such that $\mathbf{y} \neq \mathbf{0}$ but $\mathbf{x}\cdot \mathbf{y} = 0$.  Is this also true if $k = 1$?
\end{problem}
\begin{proof}
If $\mathbf{x} = \mathbf{0}$, then take $\mathbf{y}$ to be any nonzero vector in $\mathbb{R}^k$.  Otherwise, if $\mathbf{x} \neq \mathbf{0}$ there is an integer $\ell$ satisfying $1\leq \ell \leq k$ such that $x_{\ell} \neq 0$.  Choose an integer $j \neq \ell$ that satisfies $1\leq j \leq k$ (this is possible, since $k \geq 2$).  Then let $\mathbf{y}$ be the vector defined as
\begin{equation}
y_j = x_\ell, \hspace{5mm} y_\ell = - x_j, \hspace{5mm} y_i = 0 \hspace{2mm} \text{for} \  i\neq \ell,j
\end{equation}
$y_j = x_\ell \neq 0$ implies that $\mathbf{y} \neq \mathbf{0}$ and the inner product $\mathbf{x} \cdot \mathbf{y}$ satisfies:
\begin{equation}
\sum_{i = 1}^k y_i x_i = y_j x_j + y_\ell x_\ell = x_\ell x_j - x_j x_\ell = 0
\end{equation}
When $k = 1$, the inner product corresponds to standard scalar multiplication.  The result fails for $x = 1$, because if $xy = 0$, we have:
\begin{equation}
0 = xy = 1y = y
\end{equation}
by the multiplication axioms for a field.  Thus the result does not hold in $\mathbb{R}^1$.
\end{proof}
\begin{problem}{19}

\end{problem}
\begin{problem}{20}

\end{problem}
\end{document}